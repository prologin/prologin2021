% This file was generated by stechec2-generator. DO NOT EDIT.

\noindent \begin{tabular}{lp{11cm}}
\textbf{Constante:} & NB\_TOURS \\
\textbf{Valeur:} & 200 \\
\textbf{Description:} & Nombre de tours à jouer avant la fin de la partie. \\
\end{tabular}
\vspace{0.2cm} \\

\noindent \begin{tabular}{lp{11cm}}
\textbf{Constante:} & NB\_PANDAS \\
\textbf{Valeur:} & 2 \\
\textbf{Description:} & Nombre de pandas par joueur. \\
\end{tabular}
\vspace{0.2cm} \\

\noindent \begin{tabular}{lp{11cm}}
\textbf{Constante:} & NB\_TOURS\_PERTE\_BEBE \\
\textbf{Valeur:} & 3 \\
\textbf{Description:} & Nombre de tours nécessaires pour faire tomber un bébé panda. \\
\end{tabular}
\vspace{0.2cm} \\

\noindent \begin{tabular}{lp{11cm}}
\textbf{Constante:} & VALEUR\_MAX\_PONT \\
\textbf{Valeur:} & 6 \\
\textbf{Description:} & Valeur max d'un pont (les valeurs sont comprises entre 1 et cette constante inclus). \\
\end{tabular}
\vspace{0.2cm} \\

\noindent \begin{tabular}{lp{11cm}}
\textbf{Constante:} & NB\_POINTS\_CAPTURE\_BEBE \\
\textbf{Valeur:} & 10 \\
\textbf{Description:} & Nombre de points obtenus à la capture d'un bébé pandas. \\
\end{tabular}
\vspace{0.2cm} \\


\functitle{case\_type} \\
\noindent
\begin{tabular}[t]{@{\extracolsep{0pt}}>{\bfseries}lp{10cm}}
Description~: & Types de cases \\
Valeurs~: &
\small
\begin{tabular}[t]{@{\extracolsep{0pt}}lp{7cm}}
    \textsl{LIBRE}~: & Case libre \\
    \textsl{OBSTACLE}~: & Obstacle \\
    \textsl{PONT}~: & Pont \\
    \textsl{BEBE}~: & Bébé panda \\
\end{tabular} \\
\end{tabular}

\functitle{direction} \\
\noindent
\begin{tabular}[t]{@{\extracolsep{0pt}}>{\bfseries}lp{10cm}}
Description~: & Directions cardinales \\
Valeurs~: &
\small
\begin{tabular}[t]{@{\extracolsep{0pt}}lp{7cm}}
    \textsl{NORD\_EST}~: & Direction : nord-est \\
    \textsl{SUD\_EST}~: & Direction : sud-est \\
    \textsl{SUD}~: & Direction : sud \\
    \textsl{SUD\_OUEST}~: & Direction : sud-ouest \\
    \textsl{NORD\_OUEST}~: & Direction : nord-ouest \\
    \textsl{NORD}~: & Direction : nord \\
\end{tabular} \\
\end{tabular}

\functitle{erreur} \\
\noindent
\begin{tabular}[t]{@{\extracolsep{0pt}}>{\bfseries}lp{10cm}}
Description~: & Erreurs possibles \\
Valeurs~: &
\small
\begin{tabular}[t]{@{\extracolsep{0pt}}lp{7cm}}
    \textsl{OK}~: & L'action s'est effectuée avec succès. \\
    \textsl{POSITION\_INVALIDE}~: & La position spécifiée n'est pas sur la rivière. \\
    \textsl{POSITION\_OBSTACLE}~: & La position spécifiée est un obstacle. \\
    \textsl{MAUVAIS\_NOMBRE}~: & La hauteur de la position spécifiée ne correspond pas. \\
    \textsl{DEPLACEMENT\_HORS\_LIMITES}~: & Ce déplacement fait sortir un panda des limites de la rivière. \\
    \textsl{DIRECTION\_INVALIDE}~: & La direction spécifiée n'existe pas. \\
    \textsl{MOUVEMENT\_INVALIDE}~: & Le panda ne peut pas se déplacer dans cette direction. \\
    \textsl{POSE\_INVALIDE}~: & Le pont ne peut pas être placé a cette position et dans cette direction. \\
    \textsl{ID\_PANDA\_INVALIDE}~: & Le panda spécifié n'existe pas. \\
    \textsl{ACTION\_DEJA\_EFFECTUEE}~: & Une action a déjà été effectuée ce tour. \\
    \textsl{DRAPEAU\_INVALIDE}~: & Le drapeau spécifié n'existe pas. \\
    \textsl{DEPLACEMENT\_EN\_ARRIERE}~: & La panda c'est déjà déplacé sur cette case. \\
\end{tabular} \\
\end{tabular}

\functitle{action\_type} \\
\noindent
\begin{tabular}[t]{@{\extracolsep{0pt}}>{\bfseries}lp{10cm}}
Description~: & Types d'actions \\
Valeurs~: &
\small
\begin{tabular}[t]{@{\extracolsep{0pt}}lp{7cm}}
    \textsl{ACTION\_DEPLACER}~: & Action ``deplacer``. \\
    \textsl{ACTION\_POSER}~: & Action ``poser``. \\
\end{tabular} \\
\end{tabular}

\functitle{debug\_drapeau} \\
\noindent
\begin{tabular}[t]{@{\extracolsep{0pt}}>{\bfseries}lp{10cm}}
Description~: & Types de drapeau de débug \\
Valeurs~: &
\small
\begin{tabular}[t]{@{\extracolsep{0pt}}lp{7cm}}
    \textsl{AUCUN\_DRAPEAU}~: & Aucun drapeau, enlève le drapeau présent \\
    \textsl{DRAPEAU\_BLEU}~: & Drapeau bleu \\
    \textsl{DRAPEAU\_VERT}~: & Drapeau vert \\
    \textsl{DRAPEAU\_ROUGE}~: & Drapeau rouge \\
\end{tabular} \\
\end{tabular}



\functitle{position}
\begin{lst-c++}
struct position {
    int x;
    int y;
};
\end{lst-c++}
\noindent
\begin{tabular}[t]{@{\extracolsep{0pt}}>{\bfseries}lp{10cm}}
Description~: & Position du panda. \\
Champs~: &
\small
\begin{tabular}[t]{@{\extracolsep{0pt}}lp{7cm}}
    \textsl{x}~: & Coordonnée : x \\
    \textsl{y}~: & Coordonnée : y \\
\end{tabular} \\
\end{tabular}

\functitle{pont\_type}
\begin{lst-c++}
struct pont\_type {
    position debut_pos;
    position fin_pos;
    int debut_val;
    int fin_val;
};
\end{lst-c++}
\noindent
\begin{tabular}[t]{@{\extracolsep{0pt}}>{\bfseries}lp{10cm}}
Description~: & Case type pont, contient la case de début et de fin. La case de début a une valeur qui s'incrémente, et celle de fin se decrémente. \\
Champs~: &
\small
\begin{tabular}[t]{@{\extracolsep{0pt}}lp{7cm}}
    \textsl{debut\_pos}~: & Position de la case de début \\
    \textsl{fin\_pos}~: & Position de la case de fin \\
    \textsl{debut\_val}~: & Valeur de la case de début \\
    \textsl{fin\_val}~: & Valeur de la case de début \\
\end{tabular} \\
\end{tabular}

\functitle{panda\_info}
\begin{lst-c++}
struct panda\_info {
    position panda_pos;
    int id_joueur;
    int id_panda;
    int num_bebes;
};
\end{lst-c++}
\noindent
\begin{tabular}[t]{@{\extracolsep{0pt}}>{\bfseries}lp{10cm}}
Description~: & Panda et son joueur \\
Champs~: &
\small
\begin{tabular}[t]{@{\extracolsep{0pt}}lp{7cm}}
    \textsl{panda\_pos}~: & Position du panda \\
    \textsl{id\_joueur}~: & Identifiant du joueur qui contrôle le panda \\
    \textsl{id\_panda}~: & Identifiant du panda relatif au joueur \\
    \textsl{num\_bebes}~: & Nombre de bébés qui sont portés par le panda parent \\
\end{tabular} \\
\end{tabular}

\functitle{bebe\_info}
\begin{lst-c++}
struct bebe\_info {
    position bebe_pos;
    int id_bebe_joueur;
};
\end{lst-c++}
\noindent
\begin{tabular}[t]{@{\extracolsep{0pt}}>{\bfseries}lp{10cm}}
Description~: & Bébé panda à ramener. \\
Champs~: &
\small
\begin{tabular}[t]{@{\extracolsep{0pt}}lp{7cm}}
    \textsl{bebe\_pos}~: & Position du bébé panda \\
    \textsl{id\_bebe\_joueur}~: & Identifiant du joueur qui peut saver le bébé \\
\end{tabular} \\
\end{tabular}

\functitle{tour\_info}
\begin{lst-c++}
struct tour\_info {
    int id_panda_joue;
    int id_tour;
};
\end{lst-c++}
\noindent
\begin{tabular}[t]{@{\extracolsep{0pt}}>{\bfseries}lp{10cm}}
Description~: & Information sur un tour particulier. \\
Champs~: &
\small
\begin{tabular}[t]{@{\extracolsep{0pt}}lp{7cm}}
    \textsl{id\_panda\_joue}~: & Identifiant du panda qui joue \\
    \textsl{id\_tour}~: & Identifiant unique du tour (compteur) \\
\end{tabular} \\
\end{tabular}

\functitle{carte\_info}
\begin{lst-c++}
struct carte\_info {
    int taille_x;
    int taille_y;
};
\end{lst-c++}
\noindent
\begin{tabular}[t]{@{\extracolsep{0pt}}>{\bfseries}lp{10cm}}
Description~: & Information sur la carte de la partie en cours. \\
Champs~: &
\small
\begin{tabular}[t]{@{\extracolsep{0pt}}lp{7cm}}
    \textsl{taille\_x}~: & La taille de la carte pour les coordonnées x {[}0; taille\_x{[} \\
    \textsl{taille\_y}~: & La taille de la carte pour les coordonnées y {[}0; taille\_y{[} \\
\end{tabular} \\
\end{tabular}

\functitle{action\_hist}
\begin{lst-c++}
struct action\_hist {
    action_type type_action;
    int action_id_panda;
    direction dir;
    int valeur_debut;
    int valeur_fin;
    position pos_debut;
    position pos_fin;
};
\end{lst-c++}
\noindent
\begin{tabular}[t]{@{\extracolsep{0pt}}>{\bfseries}lp{10cm}}
Description~: & Action représentée dans l'historique. \\
Champs~: &
\small
\begin{tabular}[t]{@{\extracolsep{0pt}}lp{7cm}}
    \textsl{type\_action}~: & Type de l'action \\
    \textsl{action\_id\_panda}~: & Identifiant du panda concerné par l'action \\
    \textsl{dir}~: & Direction visée par le panda durant le déplacement \\
    \textsl{valeur\_debut}~: & Valeur au début du pont posé (de 1 à 6 inclus) \\
    \textsl{valeur\_fin}~: & Valeur à la fin du pont posé (de 1 à 6 inclus) \\
    \textsl{pos\_debut}~: & Position du début du pont posé \\
    \textsl{pos\_fin}~: & Position de la fin du pont posé \\
\end{tabular} \\
\end{tabular}



\begin{minipage}{\linewidth}
\functitle{deplacer}
\begin{lst-c++}
erreur deplacer(direction dir)
\end{lst-c++}
\noindent
\begin{tabular}[t]{@{\extracolsep{0pt}}>{\bfseries}lp{10cm}}
Description~: & Déplace le panda ``id\_panda`` sur le pont choisi. \\
Paramètres~: &
\begin{tabular}[t]{@{\extracolsep{0pt}}ll}
    \textsl{dir}~: & Direction visée \\
  \end{tabular} \\
\end{tabular} \\[0.3cm]
\end{minipage}

\begin{minipage}{\linewidth}
\functitle{poser}
\begin{lst-c++}
erreur poser(position position_debut, direction dir, int pont_debut, int pont_fin)
\end{lst-c++}
\noindent
\begin{tabular}[t]{@{\extracolsep{0pt}}>{\bfseries}lp{10cm}}
Description~: & Pose un pont dans la direction choisie à partir du panda ``id\_panda``. \\
Paramètres~: &
\begin{tabular}[t]{@{\extracolsep{0pt}}ll}
    \textsl{position\_debut}~: & Position de début du pont \\
    \textsl{dir}~: & Direction visée \\
    \textsl{pont\_debut}~: & Début du pont placé \\
    \textsl{pont\_fin}~: & Fin du pont \\
  \end{tabular} \\
\end{tabular} \\[0.3cm]
\end{minipage}

\begin{minipage}{\linewidth}
\functitle{debug\_afficher\_drapeau}
\begin{lst-c++}
erreur debug_afficher_drapeau(position pos, debug_drapeau drapeau)
\end{lst-c++}
\noindent
\begin{tabular}[t]{@{\extracolsep{0pt}}>{\bfseries}lp{10cm}}
Description~: & Affiche le drapeau spécifié sur la case indiquée \\
Paramètres~: &
\begin{tabular}[t]{@{\extracolsep{0pt}}ll}
    \textsl{pos}~: & Case ciblée \\
    \textsl{drapeau}~: & Drapeau à afficher sur la case \\
  \end{tabular} \\
\end{tabular} \\[0.3cm]
\end{minipage}

\begin{minipage}{\linewidth}
\functitle{type\_case}
\begin{lst-c++}
case_type type_case(position pos)
\end{lst-c++}
\noindent
\begin{tabular}[t]{@{\extracolsep{0pt}}>{\bfseries}lp{10cm}}
Description~: & Renvoie le type d'une case donnée. \\
Paramètres~: &
\begin{tabular}[t]{@{\extracolsep{0pt}}ll}
    \textsl{pos}~: & Case choisie \\
  \end{tabular} \\
\end{tabular} \\[0.3cm]
\end{minipage}

\begin{minipage}{\linewidth}
\functitle{panda\_sur\_case}
\begin{lst-c++}
int panda_sur_case(position pos)
\end{lst-c++}
\noindent
\begin{tabular}[t]{@{\extracolsep{0pt}}>{\bfseries}lp{10cm}}
Description~: & Renvoie le numéro du joueur à qui appartient panda sur la case indiquée. Renvoie -1 s'il n'y a pas de panda ou si la position est invalide. \\
Paramètres~: &
\begin{tabular}[t]{@{\extracolsep{0pt}}ll}
    \textsl{pos}~: & Case choisie \\
  \end{tabular} \\
\end{tabular} \\[0.3cm]
\end{minipage}

\begin{minipage}{\linewidth}
\functitle{bebe\_panda\_sur\_case}
\begin{lst-c++}
int bebe_panda_sur_case(position pos)
\end{lst-c++}
\noindent
\begin{tabular}[t]{@{\extracolsep{0pt}}>{\bfseries}lp{10cm}}
Description~: & Renvoie le numéro du joueur à qui appartient le bébé panda sur la case indiquée. Renvoie -1 s'il n'y a pas de bébé panda ou si la position est invalide. \\
Paramètres~: &
\begin{tabular}[t]{@{\extracolsep{0pt}}ll}
    \textsl{pos}~: & Case choisie \\
  \end{tabular} \\
\end{tabular} \\[0.3cm]
\end{minipage}

\begin{minipage}{\linewidth}
\functitle{position\_panda}
\begin{lst-c++}
position position_panda(int id_joueur, int id_panda)
\end{lst-c++}
\noindent
\begin{tabular}[t]{@{\extracolsep{0pt}}>{\bfseries}lp{10cm}}
Description~: & Indique la position du panda sur la rivière désigné par le numéro ``id\_panda`` appartenant au joueur ``id\_joueur``. Si la description du panda est incorrecte, la position (-1, -1) est renvoyée. \\
Paramètres~: &
\begin{tabular}[t]{@{\extracolsep{0pt}}ll}
    \textsl{id\_joueur}~: & Numéro du joueur \\
    \textsl{id\_panda}~: & Numéro du panda \\
  \end{tabular} \\
\end{tabular} \\[0.3cm]
\end{minipage}

\begin{minipage}{\linewidth}
\functitle{info\_pont}
\begin{lst-c++}
pont_type info_pont(position pos)
\end{lst-c++}
\noindent
\begin{tabular}[t]{@{\extracolsep{0pt}}>{\bfseries}lp{10cm}}
Description~: & Renvoie les informations relatives au pont situé à cette position. Le pont est constitué de deux cases. Si aucun pont n'est placé à cette position ou si la position est invalide, les membres debut\_val et fin\_val de la structure ``pont\_type`` renvoyée sont initialisés à -1. \\
Paramètres~: &
\begin{tabular}[t]{@{\extracolsep{0pt}}ll}
    \textsl{pos}~: & Case choisie \\
  \end{tabular} \\
\end{tabular} \\[0.3cm]
\end{minipage}

\begin{minipage}{\linewidth}
\functitle{info\_panda}
\begin{lst-c++}
panda_info info_panda(position pos)
\end{lst-c++}
\noindent
\begin{tabular}[t]{@{\extracolsep{0pt}}>{\bfseries}lp{10cm}}
Description~: & Renvoie la description d'un panda en fonction d'une position donnée. Si le panda n'est pas présent sur la carte, ou si la position est invalide, tous les membres de la structure ``panda\_info`` renvoyée sont initialisés à -1. \\
Paramètres~: &
\begin{tabular}[t]{@{\extracolsep{0pt}}ll}
    \textsl{pos}~: & Case choisie \\
  \end{tabular} \\
\end{tabular} \\[0.3cm]
\end{minipage}

\begin{minipage}{\linewidth}
\functitle{liste\_pandas}
\begin{lst-c++}
panda_info array liste_pandas()
\end{lst-c++}
\noindent
\begin{tabular}[t]{@{\extracolsep{0pt}}>{\bfseries}lp{10cm}}
Description~: & Renvoie la liste de tous les pandas présents durant la partie. \\
\end{tabular} \\[0.3cm]
\end{minipage}

\begin{minipage}{\linewidth}
\functitle{liste\_bebes}
\begin{lst-c++}
bebe_info array liste_bebes()
\end{lst-c++}
\noindent
\begin{tabular}[t]{@{\extracolsep{0pt}}>{\bfseries}lp{10cm}}
Description~: & Renvoie la liste de tous les bébés présents sur la carte, et et pas encore sauvés. \\
\end{tabular} \\[0.3cm]
\end{minipage}

\begin{minipage}{\linewidth}
\functitle{positions\_adjacentes}
\begin{lst-c++}
position array positions_adjacentes(position pos)
\end{lst-c++}
\noindent
\begin{tabular}[t]{@{\extracolsep{0pt}}>{\bfseries}lp{10cm}}
Description~: & Renvoie la liste de toutes les positions adjacentes à la position donnée. \\
Paramètres~: &
\begin{tabular}[t]{@{\extracolsep{0pt}}ll}
    \textsl{pos}~: & Case choisie \\
  \end{tabular} \\
\end{tabular} \\[0.3cm]
\end{minipage}

\begin{minipage}{\linewidth}
\functitle{position\_dans\_direction}
\begin{lst-c++}
position position_dans_direction(position pos, direction dir)
\end{lst-c++}
\noindent
\begin{tabular}[t]{@{\extracolsep{0pt}}>{\bfseries}lp{10cm}}
Description~: & Renvoie la position relative à la direction donnée par rapport à une position d'origine. Si une telle position serait invalide, la position \{-1, -1\} est renvoyée. \\
Paramètres~: &
\begin{tabular}[t]{@{\extracolsep{0pt}}ll}
    \textsl{pos}~: & Position d'origine \\
    \textsl{dir}~: & Direction \\
  \end{tabular} \\
\end{tabular} \\[0.3cm]
\end{minipage}

\begin{minipage}{\linewidth}
\functitle{direction\_entre\_positions}
\begin{lst-c++}
int direction_entre_positions(position origine, position cible)
\end{lst-c++}
\noindent
\begin{tabular}[t]{@{\extracolsep{0pt}}>{\bfseries}lp{10cm}}
Description~: & Renvoie la direction telle que position\_dans\_direction(origine, cible) == direction. Si aucune telle direction n'existe, -1 est renvoyée. \\
Paramètres~: &
\begin{tabular}[t]{@{\extracolsep{0pt}}ll}
    \textsl{origine}~: & Position d'origine \\
    \textsl{cible}~: & Position relative à l'origine \\
  \end{tabular} \\
\end{tabular} \\[0.3cm]
\end{minipage}

\begin{minipage}{\linewidth}
\functitle{historique}
\begin{lst-c++}
action_hist array historique()
\end{lst-c++}
\noindent
\begin{tabular}[t]{@{\extracolsep{0pt}}>{\bfseries}lp{10cm}}
Description~: & Renvoie la liste des actions effectuées par l’adversaire durant son tour, dans l'ordre chronologique. Les actions de débug n'apparaissent pas dans cette liste. \\
\end{tabular} \\[0.3cm]
\end{minipage}

\begin{minipage}{\linewidth}
\functitle{score}
\begin{lst-c++}
int score(int id_joueur)
\end{lst-c++}
\noindent
\begin{tabular}[t]{@{\extracolsep{0pt}}>{\bfseries}lp{10cm}}
Description~: & Renvoie le score du joueur ``id\_joueur``. Renvoie -1 si le joueur est invalide. \\
Paramètres~: &
\begin{tabular}[t]{@{\extracolsep{0pt}}ll}
    \textsl{id\_joueur}~: & Numéro du joueur \\
  \end{tabular} \\
\end{tabular} \\[0.3cm]
\end{minipage}

\begin{minipage}{\linewidth}
\functitle{moi}
\begin{lst-c++}
int moi()
\end{lst-c++}
\noindent
\begin{tabular}[t]{@{\extracolsep{0pt}}>{\bfseries}lp{10cm}}
Description~: & Renvoie votre numéro de joueur. \\
\end{tabular} \\[0.3cm]
\end{minipage}

\begin{minipage}{\linewidth}
\functitle{adversaire}
\begin{lst-c++}
int adversaire()
\end{lst-c++}
\noindent
\begin{tabular}[t]{@{\extracolsep{0pt}}>{\bfseries}lp{10cm}}
Description~: & Renvoie le numéro de joueur de votre adversaire. \\
\end{tabular} \\[0.3cm]
\end{minipage}

\begin{minipage}{\linewidth}
\functitle{info\_tour}
\begin{lst-c++}
tour_info info_tour()
\end{lst-c++}
\noindent
\begin{tabular}[t]{@{\extracolsep{0pt}}>{\bfseries}lp{10cm}}
Description~: & Renvoie le tour actuel. \\
\end{tabular} \\[0.3cm]
\end{minipage}

\begin{minipage}{\linewidth}
\functitle{info\_carte}
\begin{lst-c++}
carte_info info_carte()
\end{lst-c++}
\noindent
\begin{tabular}[t]{@{\extracolsep{0pt}}>{\bfseries}lp{10cm}}
Description~: & Renvoie la carte pour la partie en cours. \\
\end{tabular} \\[0.3cm]
\end{minipage}

\begin{minipage}{\linewidth}
\functitle{afficher\_case\_type}
\begin{lst-c++}
void afficher_case_type(case_type v)
\end{lst-c++}
\noindent
\begin{tabular}[t]{@{\extracolsep{0pt}}>{\bfseries}lp{10cm}}
Description~: & Affiche le contenu d'une valeur de type case\_type \\
Paramètres~: &
\begin{tabular}[t]{@{\extracolsep{0pt}}ll}
    \textsl{v}~: & The value to display \\
  \end{tabular} \\
\end{tabular} \\[0.3cm]
\end{minipage}

\begin{minipage}{\linewidth}
\functitle{afficher\_direction}
\begin{lst-c++}
void afficher_direction(direction v)
\end{lst-c++}
\noindent
\begin{tabular}[t]{@{\extracolsep{0pt}}>{\bfseries}lp{10cm}}
Description~: & Affiche le contenu d'une valeur de type direction \\
Paramètres~: &
\begin{tabular}[t]{@{\extracolsep{0pt}}ll}
    \textsl{v}~: & The value to display \\
  \end{tabular} \\
\end{tabular} \\[0.3cm]
\end{minipage}

\begin{minipage}{\linewidth}
\functitle{afficher\_erreur}
\begin{lst-c++}
void afficher_erreur(erreur v)
\end{lst-c++}
\noindent
\begin{tabular}[t]{@{\extracolsep{0pt}}>{\bfseries}lp{10cm}}
Description~: & Affiche le contenu d'une valeur de type erreur \\
Paramètres~: &
\begin{tabular}[t]{@{\extracolsep{0pt}}ll}
    \textsl{v}~: & The value to display \\
  \end{tabular} \\
\end{tabular} \\[0.3cm]
\end{minipage}

\begin{minipage}{\linewidth}
\functitle{afficher\_action\_type}
\begin{lst-c++}
void afficher_action_type(action_type v)
\end{lst-c++}
\noindent
\begin{tabular}[t]{@{\extracolsep{0pt}}>{\bfseries}lp{10cm}}
Description~: & Affiche le contenu d'une valeur de type action\_type \\
Paramètres~: &
\begin{tabular}[t]{@{\extracolsep{0pt}}ll}
    \textsl{v}~: & The value to display \\
  \end{tabular} \\
\end{tabular} \\[0.3cm]
\end{minipage}

\begin{minipage}{\linewidth}
\functitle{afficher\_debug\_drapeau}
\begin{lst-c++}
void afficher_debug_drapeau(debug_drapeau v)
\end{lst-c++}
\noindent
\begin{tabular}[t]{@{\extracolsep{0pt}}>{\bfseries}lp{10cm}}
Description~: & Affiche le contenu d'une valeur de type debug\_drapeau \\
Paramètres~: &
\begin{tabular}[t]{@{\extracolsep{0pt}}ll}
    \textsl{v}~: & The value to display \\
  \end{tabular} \\
\end{tabular} \\[0.3cm]
\end{minipage}

\begin{minipage}{\linewidth}
\functitle{afficher\_position}
\begin{lst-c++}
void afficher_position(position v)
\end{lst-c++}
\noindent
\begin{tabular}[t]{@{\extracolsep{0pt}}>{\bfseries}lp{10cm}}
Description~: & Affiche le contenu d'une valeur de type position \\
Paramètres~: &
\begin{tabular}[t]{@{\extracolsep{0pt}}ll}
    \textsl{v}~: & The value to display \\
  \end{tabular} \\
\end{tabular} \\[0.3cm]
\end{minipage}

\begin{minipage}{\linewidth}
\functitle{afficher\_pont\_type}
\begin{lst-c++}
void afficher_pont_type(pont_type v)
\end{lst-c++}
\noindent
\begin{tabular}[t]{@{\extracolsep{0pt}}>{\bfseries}lp{10cm}}
Description~: & Affiche le contenu d'une valeur de type pont\_type \\
Paramètres~: &
\begin{tabular}[t]{@{\extracolsep{0pt}}ll}
    \textsl{v}~: & The value to display \\
  \end{tabular} \\
\end{tabular} \\[0.3cm]
\end{minipage}

\begin{minipage}{\linewidth}
\functitle{afficher\_panda\_info}
\begin{lst-c++}
void afficher_panda_info(panda_info v)
\end{lst-c++}
\noindent
\begin{tabular}[t]{@{\extracolsep{0pt}}>{\bfseries}lp{10cm}}
Description~: & Affiche le contenu d'une valeur de type panda\_info \\
Paramètres~: &
\begin{tabular}[t]{@{\extracolsep{0pt}}ll}
    \textsl{v}~: & The value to display \\
  \end{tabular} \\
\end{tabular} \\[0.3cm]
\end{minipage}

\begin{minipage}{\linewidth}
\functitle{afficher\_bebe\_info}
\begin{lst-c++}
void afficher_bebe_info(bebe_info v)
\end{lst-c++}
\noindent
\begin{tabular}[t]{@{\extracolsep{0pt}}>{\bfseries}lp{10cm}}
Description~: & Affiche le contenu d'une valeur de type bebe\_info \\
Paramètres~: &
\begin{tabular}[t]{@{\extracolsep{0pt}}ll}
    \textsl{v}~: & The value to display \\
  \end{tabular} \\
\end{tabular} \\[0.3cm]
\end{minipage}

\begin{minipage}{\linewidth}
\functitle{afficher\_tour\_info}
\begin{lst-c++}
void afficher_tour_info(tour_info v)
\end{lst-c++}
\noindent
\begin{tabular}[t]{@{\extracolsep{0pt}}>{\bfseries}lp{10cm}}
Description~: & Affiche le contenu d'une valeur de type tour\_info \\
Paramètres~: &
\begin{tabular}[t]{@{\extracolsep{0pt}}ll}
    \textsl{v}~: & The value to display \\
  \end{tabular} \\
\end{tabular} \\[0.3cm]
\end{minipage}

\begin{minipage}{\linewidth}
\functitle{afficher\_carte\_info}
\begin{lst-c++}
void afficher_carte_info(carte_info v)
\end{lst-c++}
\noindent
\begin{tabular}[t]{@{\extracolsep{0pt}}>{\bfseries}lp{10cm}}
Description~: & Affiche le contenu d'une valeur de type carte\_info \\
Paramètres~: &
\begin{tabular}[t]{@{\extracolsep{0pt}}ll}
    \textsl{v}~: & The value to display \\
  \end{tabular} \\
\end{tabular} \\[0.3cm]
\end{minipage}

\begin{minipage}{\linewidth}
\functitle{afficher\_action\_hist}
\begin{lst-c++}
void afficher_action_hist(action_hist v)
\end{lst-c++}
\noindent
\begin{tabular}[t]{@{\extracolsep{0pt}}>{\bfseries}lp{10cm}}
Description~: & Affiche le contenu d'une valeur de type action\_hist \\
Paramètres~: &
\begin{tabular}[t]{@{\extracolsep{0pt}}ll}
    \textsl{v}~: & The value to display \\
  \end{tabular} \\
\end{tabular} \\[0.3cm]
\end{minipage}